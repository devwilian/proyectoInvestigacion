\chapter*{Resumen}
Hoy en día el papel que cumple las tecnologías de información (TI) dentro de las organizaciones, la manera como las instituciones dependen de las herramientas y servicios que ofrecen las TI para manejar cantidades asombrosas de información y como los procesos de la organización son cada vez más automatizados, nos hace ver que día a día las TIC son mucho más importante de lo imaginado.\\ 
Sin embargo, la dependencia de las tecnologías de información dentro de las actividades de una organización en nuestro caso de estudio la Universidad Nacional San Antonio Abad del Cusco (UNSAAC), podrían resultar ser contraproducente. La dirección de Sistemas de Información con su unidad de Red de Comunicaciones (RCU), unidad de Centro de Cómputo (CC), unidad y biblioteca y unidad de estadística, es la encargada de brindar los servicios de TI, pero que no son ajenos a sufrir falencias, cuando se produce una interrupción de algún sistema por la falla de algún proceso, el corte abrupto del servicio de internet en la organización por la falla de los equipos de conexión, la falta de personal en el área de TI o personal poco capacitado, por mencionar algunos de los inconvenientes podría generar pérdidas e insatisfacción en los usuarios. A medida que el área de TI de la organización tenga implementada un modelo de gestión de servicios de TI las falencias se reducirían significativamente, pero no es el Caso de la dirección de sistemas de información de la UNSAAC, el no contar con un modelo de gestión de servicios de TI, hace que resolver un inconveniente genere pérdidas aún peores por no contar con los procedimientos y documentación adecuados.\\
Es por eso surge la necesidad de diseñar, un modelo de gestión de servicios de TI dentro de la dirección de sistemas de información de la Universidad Nacional de San Antonio Abad del Cusco (UNSAAC), haciendo uso de los marcos de referencia COBIT e ITIL para comprender la correcta gestión de los procesos, el buen manejo del recurso humano, la correcta administración tecnológica, desarrollar guías claras y buenas prácticas para el control de las TI, con el objetivo de que la UNSAAC  pueda mejorar la calidad de los servicios, lograr alcanzar sus metas con un adecuado control.
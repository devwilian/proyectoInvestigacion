\chapter{Aspectos Generales}
\section{Antecedentes y estado del arte}
\subsection{Estado del arte}
Después de la revisión de la bibliografía existente se ve que actualmente la Dirección de Sistemas de Información de la UNSAAC no cuenta con los procesos bien definidos, ni mucho menos un enfoque de gestión de servicios según los marcos de trabajo COBIT e ITIL, pero existen trabajos relacionados que buscan diseñar modelos para la mejora de gestión de servicios en las áreas de TI en organizaciones que ya tienen sus procesos de TI bien definidos.
\subsection{Antecedentes}
\subsubsection{Antecedentes locales}
\textbf{``Propuesta de implementación para la reestructuración orgánica de la dirección de sistemas de información de la UNSAAC''},Br. Manuel Moisés Arizabal Vera, Universidad Nacional de San Antonio Abad del Cusco, 2019.\\[2pt]
Se propuso cambios en la estructura de la Dirección de Tecnologías de la Información (actual Dirección de Sistemas de Información) como la creación de las Unidades Funcionales de: Infraestructura Tecnológica, Sistemas de Información, Soporte y Operación de Infraestructura Tecnológica, Gobierno de Tecnológicas de la Información y Seguridad de la Información. también se implementó el Manual de Obligaciones y Funciones para con la empresa.\\

\textbf{``Propuesta de la implementación de un centro de servicios de la tecnología de la información al usuario basado en ITIL v3 para la dirección de telecomunicaciones y tecnologías de la información de la Universidad Andina del Cusco''}, Sota Orellana, Luis Alberto Peña Carlos, Shirley Liliana Fuentes Vargas, Diana Melanie; Universidad Andina del Cusco. (2016.\\[2pt]
En este trabajo se realizó el diseño de un centro de servicio (Service Desk), cuyo objetivo principal es proveer un único punto de contacto para dar asesoramiento y soporte técnico a las solicitudes de los servicios de los usuarios. Se especificó los roles, responsabilidades y perfiles de personal de centro de servicios de TI; se elaboró los prototipos del Acuerdo de Nivel de Servicios (SLA’s) y el Catálogo de Servicios; se elaboró los diagramas de actividad de los procesos en la gestión de Petición de servicio, Incidencias y Problemas basado en ITIL V3 y finalmente se desarrollaron las conclusiones y recomendaciones.\\

\textbf{``Diseño de una base de datos de gestión de la configuración basado en los procesos de gestión de la configuración y activos según ITIL 2011, ISO/IEC 20000 Y COBIT 5 para una empresa de aluminios y vidrios''}, Carlos Alberto Bustamante Aponte, Pontificia Universidad Católica del Perú, 2016.\\[2pt]
Este proyecto tiene como finalidad establecer una guía de buenas prácticas en el área de TI, la cual pueda ser consultada al realizarse las tareas que involucran los procesos de gestión de configuraciones y gestión de cambios. también facilitó los procesos de gestión de la configuración gracias a los lineamientos de ITIL 2011, ISO/IEC 20000 y Cobit 5.

\subsubsection{Antecedentes nacionales}
\textbf{``Implantación de los procesos de gestión de incidentes y gestión de problemas según ITIL v3.0 en el área de tecnologías de información de una entidad financiera''}, Jesús Rafael Gómez Álvarez, Pontificia Universidad Católica del Perú, 2012\\[2pt]
En este proyecto se define los procesos de gestión de incidentes y problemas con una visión de organización para la atención de estos eventos según el marco de referencia ITIL. El avance de la implementación de este proyecto se muestra mes a mes.\\

\textbf{``Implantación de los procesos de gestión de incidentes y gestión de problemas según ITIL v3 en el área de tecnología de información de la gerencia regional de transportes y comunicaciones''} Abrahan Bernardo Garcia Alarcon, Universidad Señor de Sipán, 2016.\\[2pt]
En este proyecto se ha propuesto la implantación del marco metodológico ITIL y la utilización de un aplicativo Help desk, lo cual me permitirá identificar, evaluar, controlar y monitorear todas las incidencias y problemas que se den, para aplicar las posibles soluciones inmediatamente.\\

\textbf{``Diseño de un modelo de gestión de incidentes y gestión de problemas según ITIL v3 para mejorar el proceso de gestión de infraestructura tecnológica de la empresa Distribuciones M. OLANO S.A.C. - 2016''} Mio Gallegos Paula del Milagro, Universidad Nacional Pedro Ruiz Gallo, 2016.\\[2pt]
Se propuso un modelo en el que se ha hecho un análisis de la situación actual de sus procesos de la empresa, se ha rediseñado la Gestión de Incidentes, se ha planteado el diseño de la Gestión de Problemas y se ha realizado una rigurosa evaluación para la selección de la herramienta de software que soporte sus procesos; todo esto se hizo siguiendo los lineamientos dados por ITIL, esta propuesta contribuyó a la mejora del proceso de Gestión de Infraestructura Tecnológica en la organización.

\subsubsection{Antecedentes internacionales}
\textbf{``Implementation of incident management for data services using ITIL V3 in telecommunication operator company''}, Nugraha, A. D., \& Legowo, N.. International Conference on Applied Computer and Communication Technologies, 2017.\\[2pt]
Este proyecto tuvo un aporte en el tráfico de datos en indonesia ya que se pudo solucionar los incidentes y problemas gracias a los marcos de referencia de ITIL. Con el paso del tiempo los incidentes fueron solucionados según el nivel de emergencia y gravedad crítica.\\

\textbf{``Strategies to Improve Human Resource Management using COBIT 5 For Data and Information Centre of Ministry of Agriculture of Indonesia of Republic''}, Fitroh, Damanik, A., \& Firmansyah, A. F. 2018.\\[2pt]
Para medir la gestión de los recursos humanos en Pusdatin se puede hacer usando el marco COBIT 5. Esto se debe a que las sugerencias de COBIT son genéricas y especialmente útiles para la gestión del rendimiento del programa. Las organizaciones se benefician de la implementación de medidas marco COBIT 5 en su gobierno de TI. En relación con esto, es necesario conocer la estrategia para mejorar la gestión de los recursos humanos utilizando el marco COBIT 5.\\

\textbf{``Propuesta de modelo de evaluación de herramientas para la gestión del proceso de gestión de problemas de ITIL''}, Roig-Ferriol y José Manuel, UPV, 2015.\\[2pt]
Este trabajo se presenta una propuesta de modelo de evaluación de herramientas de soporte a uno de los principales procesos de ITIL, como es el proceso de Gestión de problema. además, este artículo se centra en ITIL debido a que es el estándar más ampliamente conocido para la gestión de Servicios de TI. Mediante ITIL se puede optimizar la gestión de los Servicios, lo cual permite un alto nivel de disponibilidad de los mismos y mejorar el grado de satisfacción de clientes y de los propios empleados de la organización.

\section{Problema de Investigacion}
\subsection{Descripcion del problema}
\subsection{Formulacion del problema}
?`Como influye el diseño de un modelo de gestion de servicios de TI segun COBIT e ITIL en la gestion de servicios de TI en la direccion de Sistemas de Informacion de la UNSAAC?
\section{Justificacion}
Desde los inicios de los tiempos el ser humano buscó la forma de estar por delante de su competencia, es por ello que ha adoptado las mejores metodologías acorde a su realidad, en los últimos años el agregar las tecnologías de información en sus procesos fueron la clave del éxito de las grandes empresas, a causa de ello la gestión de servicios de TI se ha tornado complicado al inicio, pero gracias a los diferentes marcos de trabajo; en nuestro caso de estudio, el no contar con una documentación adecuada hace que la gestión de servicios sean muy confuso y es necesario tener bien definidos los servicios y cuál es su procedimiento para que un proceso sea ejecutado de mejor manera.
El presente proyecto tiene la finalidad de diseñar un modelo de gestión de servicios de TI para que se maximice la eficacia y eficiencia.

\section{Objetivos}
\subsection{Objetivo general}
Diseñar un modelo para la gestión de servicios de TI para la Dirección de Sistemas de Información de la UNSAAC según COBIT e ITIL.
\subsection{Objetivos especificos}
\begin{itemize}
\item Identificar y describir los procesos de gestión de TI actuales de la Dirección de Sistemas de Información de la UNSAAC.
\item Elaborar modelos de negocio de los procesos de gestión de TI que se obtiene luego de identificarlos.
\item Implementar el modelo de gestión de TI a la Unidad de Red de Comunicaciones.
\end{itemize}
\section{Hipotesis}
\section{Alcances y limitaciones}
\subsection{Alcances}
\subsection{Limitaciones}
